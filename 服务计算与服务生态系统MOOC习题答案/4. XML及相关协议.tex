\subsubsection*{\S\ XML及相关协议}
\setcounter{problemname}{0}

\begin{problem}
	信息交换的范围可以分为应用内部(Intra-Application)、应用之间(Inter-Application)、系统之间(Inter-System)和公司之间(Inter-Company)。XML可以用以进行上述任意类型的信息交换。
	\hfill (\ding{51})
\end{problem}


\begin{problem}
	XML是一种以树结构交换文本信息的信息交换标准,无法被用以交换二进制数据。
	\hfill (\ding{55})
\end{problem}
\\ \begin{solution}
	XML可以用来交换二进制数据。
\end{solution}


\begin{problem}
	XML既可以面向数据,也可以面向展现。在面向展现时,XML可以使用预定好的元素和属性表达Web页面,在Web浏览器打开该页面时,如果发生结束标签缺失或标签大小写不匹配等问题时,与HTML类似,Web浏览器可以自动纠正这些错误。
	\hfill (\ding{55})
\end{problem}
\\ \begin{solution}
	XML是面向数据的,他的语义和展现无关,亦没有任何面向页面的预定义标签,任何XML均应满足格式良好的限定条件。
\end{solution}


\begin{problem}
	名称空间(Namespace),采用前缀(Prefix)$+$本地部分(Local part)表达元素和属性名称,从而解决命名冲突和全球化的问题。 通过import机制可以把隶属在不同名称空间(前缀对应不同URI)中的元素(属性)组装为一个XML文档。
	\hfill (\ding{55})
\end{problem}
\\ \begin{solution}
	无需额外机制,即可通过QName在单一XML文档中融合来源于不同Namespace的元素或者属性。
\end{solution}


\begin{problem}
	只要满足$5+1$条基本原则(单根原则、元素(Element)标签原则、元素嵌套原则、元素原则、属性原则和XML声明原则),XML文档就可以被称为良好格式化(Well-formed)。在此基础上,如有需要,还可以使用Schema脚本等对XML文档进行数据类型检查,这个过程称为验证(Validation)。
	\hfill (\ding{51})
\end{problem}

