\subsubsection*{\S\ Web Service核心}
\setcounter{problemname}{0}

\begin{problem}
	SOAP定义了在分布式异构环境中进行信息交换的消息框架和单向、无状态的通信范例。它不是一个网络传输协议,因此,必须通过绑定机制将SOAP消息与实际的网络消息格式进行关联。
	\hfill (\ding{51})
\end{problem}


\begin{problem}
	在我们使用SOAP来传递XML信息时,由上层的应用来解释SOAP所携带XML消息的语义语法。虽不是必须,一般来说,必选的核心的相关信息被放置在消息体(Body),可选的辅助性的相关信息被放置在消息头(Header)
	\hfill (\ding{51})
\end{problem}


\begin{problem}
	WSDL以XML的方式描述了一个服务的功能、调用地点和调用方式。WSDL中所定义的操作(Operation)按照风格可以分为面向文档(Document-Oriented)和面向过程(Procedure-Oriented)。其主要区别在于所使用的SOAP绑定风格不同:前者使用Message-exchange风格的SOAP消息,而后者使用RPC风格的SOAP消息。
	\hfill (\ding{55})
\end{problem}
\\ \begin{solution}
	主要的区别在于:是接受一个用XML Schema约定的XML文档作为输入输出,还是采用远端的过程接口加上映射方式决定输入输出。
\end{solution}


\begin{problem}
	WSDL模型被分为抽象部分(Abstract Section)和具体部分(Concrete Section)。抽象部分包括types、interface元素;具体部分包括binding、service元素。抽象部分和具体部分可以由不同的设计人员/团队加以定义,并加以组装。拥有相同抽象部分的服务被称为服务簇(Services Cluster),它们拥有相同/相似的服务能力,在满足其他条件(如非功能性需求)的前提下,可以相互替代。
	\hfill (\ding{51})
\end{problem}


\begin{problem}
	WSDL中的binding元素和SOAP中的绑定机制是类似的。因此,在使用WSDL+SOAP对服务进行描述时,binding元素可以省去。只有采用WSDL+HTTP对服务进行描述时,binding元素才是必须的。
	\hfill (\ding{55})
\end{problem}
\\ \begin{solution}
	WSDL中的binding元素和SOAP中的绑定机制解决不同的问题。即使是使用WSDL$+$SOAP,binding元素也是必须的。
\end{solution}

