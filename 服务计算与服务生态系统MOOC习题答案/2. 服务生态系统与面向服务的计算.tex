\subsubsection*{\S 服务生态系统与面向服务的计算}
\setcounter{problemname}{0}

\begin{problem}
	服务组合由多个装配在一起服务所构成,用以提供对业务任务或过程进行实现的功能。如果服务组合能够进一步的被封装为服务,可以认为服务组合是服务的一种实现方式。
	\hfill (\ding{51})
\end{problem}


\begin{problem}
	只要在服务库存中存在,无论是应用服务、业务服务还是编排服务,都可以作为子服务被服务组合装配。
	\hfill (\ding{51})
\end{problem}


\begin{problem}
	根据是否直接满足服务消费者的需求,可以将服务生态系统(Services ecosystem)中的服务分为垂直(Vertical)服务和水平(Horizontal)服务。因此,在进行服务系统构建的时候,可以事先将服务分划为这两种类型,并按照其特点,应用不同的设计原则,分别进行分析和设计。
	\hfill (\ding{55})
\end{problem}
\\ \begin{solution}
	分划不是排他性的,如果一个服务既能够被消费者直接调用,也能为其他服务所调用,那么该服务可以同时是垂直服务和水平服务。
\end{solution}


\begin{problem}
	由于缺乏统一的标准化组织和标准体系,目前尚没有出现成熟的面向服务编程语言,但是随着面向服务的泛型和Web Service标准的发展,面向服务编程语言有望在将来出现,并提供面向服务实现的完全支持。
	\hfill (\ding{55})
\end{problem}
\\ \begin{solution}
	面向服务着眼于对服务系统进行平台/技术/语言无关的分析和设计,不需要提供面向服务的编程语言来提供面向服务的实现。
\end{solution}


\begin{problem}
	面向服务的目标主要包括技术上的:灵活性、扩展性、鲁棒性、重用性和效率、业务需求的满足。商业上的:组织灵活性、固有的互操作性、供应商多样化选择、联盟、业务和技术一致性、削减IT投入、提高投资回报率。
	\hfill (\ding{51})
\end{problem}


\begin{problem}
	从作用域角度出发,面向服务的计算中的服务作用域往往基于一个服务生态系统;在面向对象的计算中,对象的作用域往往基于一个软件系统。
	\hfill (\ding{51})
\end{problem}


\begin{problem}
	从耦合的角度出发,在面向对象的计算中,如果我们按照功能仔细划分,达成模块和模块之间的“相对无关性”,我们可以达成与面向服务计算中类似的松耦合特性。
	\hfill (\ding{55})
\end{problem}
\\ \begin{solution}
	在面向服务计算中,松耦合还体现在运行时态动态绑定。即设计与实现分离、一个设计可以由多个实现者加以提供。
\end{solution}


\begin{problem}
	‌服务组合提供了复用性和灵活性。但是由于特定子服务可以同时参与到多个业务流程,“瓶颈效应”和“单点失效”的问题使得以服务组合方式实现的业务功能可靠性差于采用面向对象实现的类似功能。
	\hfill (\ding{55})
\end{problem}
\\ \begin{solution}
	“瓶颈效应”和“单点失效”问题可以通过提供服务的多个备选实现加以解决。
\end{solution}


\begin{problem}
	在面向对象的计算中,代码复用通过类成员的继承和库函数加以实现。面向服务的计算中,代码在服务层次复用。
	\hfill (\ding{51})
\end{problem}


\begin{problem}
	由于服务是一个网络构件,采用面向服务的计算所构造出的服务系统,天然的是一个网络应用。
	\hfill (\ding{51})
\end{problem}


