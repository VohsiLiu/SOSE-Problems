\subsubsection*{\S\ Web Service扩展}
\setcounter{problemname}{0}

\begin{problem}
	UDDI存放的信息可以分为白页(White Pages)、黄页(Yellow Pages)和绿页(Green Pages)。在使用UDDI发布Web Services时,使用白页记录WSDL的抽象部分;使用绿页记录WSDL的具体部分;使用黄页记录服务合约中的其他信息。
	\hfill (\ding{55})
\end{problem}
\\ \begin{solution}
	白页存放基本信息、黄页存放分类信息、绿页存放技术信息。
\end{solution}


\begin{problem}
	BPEL是一种使用编排风格实现服务组合的规范。一方面BPEL要求组合成员在WSDL中以partnerLinkType元素说明它在组合中的作用和地位;另一方面使用BPEL文档描述该组合的逻辑流程。BPEL公布了服务实现的细节。
	\hfill (\ding{51})
\end{problem}

\begin{problem}
	WS-Addressing被用以完成服务实现中的寻址和消息路由。WS-Security被用以实现Web Service安全相关的非功能性需求。
	\hfill (\ding{51})
\end{problem}


\begin{problem}
	WSRF被用以完成服务相关资源的定义和管理。无论是有状态服务(Stateful)还是无状态服务(Stateless)都可以使用它来定义和管理资源。
	\hfill (\ding{55})
\end{problem}
\\ \begin{solution}
	一般认为,携有资源的服务为有状态服务。
\end{solution}


\begin{problem}
	WS-Coordination被用以支持服务之间的复杂协作,它定义了协作服务和协作上下文,只能使用原子事务(Atomic Transactions)和业务活动(Business Activity)两种协议分别完成短时间协作和长时间协作。
	\hfill (\ding{55})
\end{problem}
\\ \begin{solution}
	亦可自己扩展相关的协作协议。
\end{solution}

