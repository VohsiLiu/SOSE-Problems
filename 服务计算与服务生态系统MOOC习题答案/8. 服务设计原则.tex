\subsubsection*{\S 服务设计原则}
\setcounter{problemname}{0}

\begin{problem}
	标准化服务合约原则要求制定服务生态系统中的服务合约标准。越是标准化的服务合约越利于服务的可理解性和可复用性。此外,标准化的服务合约还倾向于产生标准粒度的调用数据,从而避免不必要的数据转换,减少服务实现的复杂度。
	\hfill (\ding{51})
\end{problem}


\begin{problem}
	服务松散耦合原则允许的积极耦合包括:逻辑—合约耦合、消费者—合约耦合。
	\hfill (\ding{51})
\end{problem}


\begin{problem}
	服务的抽象原则要求服务合约对外隐藏非必要信息。这些信息抽象包括:技术信息抽象、功能抽象、程序逻辑抽象和服务质量抽象。越是抽象的服务合约,越倾向于较高的可复用性。
	\hfill (\ding{55})
\end{problem}
\\ \begin{solution}
	完全抽象的服务合约可能需要额外的其他元信息交换渠道,反而不利于服务的可复用性。
\end{solution}


\begin{problem}
	服务的可复用性原则要求将服务生态系统中的所有服务均设计为无关服务(Agnostic service),从而确保最大程度的可复用性。
	\hfill (\ding{55})
\end{problem}
\\ \begin{solution}
	不是所有的服务均是无关服务,垂直服务在大多数情况下是和消费者有关的。
\end{solution}


\begin{problem}
	服务自治原则,将运行时自治的情况分为共享自治、服务逻辑自治和完全自治。虽然高自治的服务倾向于高性能、高可靠性和可预测性,但是由于高自治服务往往需要较多的软硬件资源,在设计服务的时候需要慎重考虑。
	\hfill (\ding{51})
\end{problem}


\begin{problem}
	服务的无状态性原则,鼓励利用调用消息或状态数据库,在必要时卸载/还原服务的状态信息,将服务的实例转换为无状态/少状态,从而减少资源的消耗,提升服务的可扩展性。
	\hfill (\ding{51})
\end{problem}


\begin{problem}
	服务的可发现性原则要求服务不仅能够被发现,而且要求服务合约所提供的信息使得服务的潜在调用者能完全了解该服务的目标、能力和限制等。良好的可发现性是服务具备较高可复用性的前提条件。
	\hfill (\ding{51})
\end{problem}


\begin{problem}
	服务的可组合性原则,要求服务应尽量设计为使用服务组合来加以实现。作为组合控制器(Composition Controller)的服务调用的组合成员(Composition Member)越多,该服务组合越复杂,担任组合控制器的服务的可组合型越好。
	\hfill (\ding{55})
\end{problem}
\\ \begin{solution}
	服务的可组合性主要面对非功能性需求,由其他服务设计原则的综合表现进行评级。
\end{solution}

