\subsubsection*{\S 面向服务的架构和Web Service}
\setcounter{problemname}{0}

\begin{problem}
	服务系统中的三要素包括:服务提供者、服务消费者和服务注册(Service Registry)。其中,服务注册通过支持服务的发布和查找,实现服务提供者和服务消费者之间的松耦合,从而实现服务系统灵活、可动态配置的特点。
	\hfill (\ding{55})
\end{problem}
\\ \begin{solution}
	服务系统中没有关于服务注册的要求。
\end{solution}


\begin{problem}
	由是否拥有中心协调者作为判断,服务组合(Composition)的方法可以分为编排(Orchestration)和编导(Choreograph)。从能力上来说,它们各有不同,在实际使用时需要根据业务场景进行选择。
	\hfill (\ding{55})
\end{problem}
\\ \begin{solution}
	从能力上来说他们是等价的。
\end{solution}


\begin{problem}
	对应服务生态系统,SOA-RA(SOA参考架构)中,我们使用垂直层来分层设计并实现直接满足消费者需求的垂直服务;我们使用水平层来分层设计并实现用以实现灵活、复用的水平服务。
	\hfill (\ding{55})
\end{problem}
\\ \begin{solution}
	SOA-RA中,水平层用以实现功能,垂直层用于提供支持和QoS。他们和服务生态系统中的垂直服务和水平服务的分划方式没有关系。
\end{solution}


\begin{problem}
	服务簇由服务层进行定义,它是一类从概念上服务于同一个业务功能的服务集合。他们可以由不同的服务提供者所发布,并在功能上可以相互替代。而业务过程层中,无论采用组合还是分解的方式,实质上都是面向服务簇来进行的。
	\hfill (\ding{51})
\end{problem}


\begin{problem}
	面向服务的架构并没有限定实现技术,常见的实现技术包括但不限于:Web Service、分布式对象、构件、RESTful Services、ESB等。
	\hfill (\ding{51})
\end{problem}


