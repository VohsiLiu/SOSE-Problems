\subsubsection*{\S 服务、服务系统与面向服务的泛型}
\setcounter{problemname}{0}


\begin{problem}
	IT服务和非IT服务都满足服务的基本要求,但他们也有区别。他们的区别包括:KPIs(关键绩效指标)不同、服务雇员不同、变化的步调不同。
	\hfill (\ding{55})
\end{problem}
\\ \begin{solution}
	他们的区别在于KPIs(关键绩效指标)不同、需求管理不同、变化的步调不同。
\end{solution}


\begin{problem}
	铁三角发布MSR7限量版耳机。在普通MSR7耳机的基础上,限量版耳机听取了消费者的建议和需求,对模具和发声单元进行了大幅改进。4月1日上市贩售,限量10000只,每只耳机均附带唯一编号的收藏证明。这个模式偏服务模式。
	\hfill (\ding{55})
\end{problem}
\\ \begin{solution}
	为制造模式。
\end{solution}


\begin{problem}
	服务模式(Service Mode)和制造模式(Manufacturing Mode)的最大差异在于:服务模式的产物是服务(Service),而制造模式的产物是货物(Goods)。服务是无形的、挥发的,并可能以消费者参与的方式定制化生产;而货物是有型的、可存储的,消费者不直接参与货物的生产过程。
	\hfill (\ding{51})
\end{problem}


\begin{problem}
	服务也可以以实物作为载体。如采用定制化的方式设计、裁剪和制作衣服。虽然衣服是有型的,但是满足了消费者的个性化定制需求,可以被认为是一个以服务为主的过程。
	\hfill (\ding{51})
\end{problem}


\begin{problem}
	可口可乐公司中的企业资源管理(ERP)偏服务模式。ERP系统是一个服务系统。
	\hfill (\ding{51})
\end{problem}


\begin{problem}
	很多行业兼具制造模式和服务模式的特点,但随着时代的发展和竞争的要求,单纯的制造模式变得越来越少、服务模式的特点变得越来越显著。
	\hfill (\ding{51})
\end{problem}



\begin{problem}
	用来支持服务模式的IT系统被称为服务系统(Services System)。由于服务模式变的越来越普及,所有的软件系统均可以使用服务系统的概念进行抽象,进一步的,可以使用面向服务的方式进行分析和设计。
	\hfill (\ding{55})
\end{problem}
\\ \begin{solution}
	不是所有的软件系统都是服务系统,也不是所有的服务系统都需要使用面向服务的方式进行分析和设计。
\end{solution}


\begin{problem}
	要满足自治、开放、自描述、与实现无关的特点,无论是本地构件还是网络构件,都能够被认为是服务。
	\hfill (\ding{55})
\end{problem}
\\ \begin{solution}
服务必须是网络构件。
\end{solution}


\begin{problem}
	面向对象的分析设计过程中间,如果将系统模块化,并采用统一的方式定义模块和模块之间的接口,如果这些接口是标准的,我们就可以把它们抽象为构件。与对象相比,构件的粒度较大,利于复用。
	\hfill (\ding{55})
\end{problem}
\\ \begin{solution}
	构件是模块化的、可部署、可替换的软件系统组成部分。
\end{solution}


\begin{problem}
	引入面向服务的泛型后,一方面,复用的范围从单个服务系统扩大到了多个服务系统;另一方面,软件的开发任务被分为服务的生产和应用的开发。这使得各个服务系统的开发工作能够有效利用现有的IT资源和人力资源。
	\hfill (\ding{51})
\end{problem}



