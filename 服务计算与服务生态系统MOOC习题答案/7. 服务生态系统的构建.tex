\subsubsection*{\S 服务生态系统的构建}
\setcounter{problemname}{0}

\begin{problem}
	面向服务分析的过程主要包括:定义流程自动化需求,识别现有的自动化系统,服务候选建模。其中,服务候选建模是面向服务分析的主要步骤,以建立服务操作候选,并将其合理分组到服务候选为目的。
	\hfill (\ding{51})
\end{problem}


\begin{problem}
	在服务建模的过程中,需要避免服务候选之间的业务逻辑相互重叠。这一原则要求在任何情况下同一个业务逻辑单元仅能用一个服务操作候选进行封装。
	\hfill (\ding{55})
\end{problem}
\\ \begin{solution}
	大多数服务候选有这样的需求,但是编排和组合服务有意设计为其他服务的组合。封装遗留系统的服务也可能违反这一要求。为了安全性和粒度需求,有时候我们也会采用冗余接口的方式加以设计。
\end{solution}


\begin{problem}
	​根据业务分析方法及其自身特点,在对服务进行设计时,可以将服务分为应用服务、以实体为核心的业务服务、以任务为核心的业务服务和编排服务。各种不同类型的服务在面向服务原则的应用上各有不同,因此需要在分析和设计时区别对待,为每种类型的服务分别制定分析和设计标准。
	\hfill (\ding{51})
\end{problem}


\begin{problem}
	在提炼和应用面向服务原则的过程中,不应局限于产生该服务的初始流程和上下文。一方面,需要综合考虑服务生态系统中所有可能复用该服务的流程和上下文;另一方面,应当充分考虑服务的进一步演化和全新流程的构建。
	\hfill (\ding{51})
\end{problem}


\begin{problem}
	以任务为核心的业务可以使用服务组合加以实现,而以实体为核心的业务服务不能使用服务组合加以实现。
	\hfill (\ding{55})
\end{problem}
\\ \begin{solution}
	两者均可。
\end{solution}


