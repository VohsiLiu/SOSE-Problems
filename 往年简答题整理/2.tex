\begin{problem}
    为何需要构建服务生态系统?什么是服务生态系统中的垂直服务和水平服务?它们有何联系和区别?试举例加以说明。
    \end{problem}
    
    \begin{solution}
    服务生态系统是一种由多个互相依赖和交互的服务组成的系统,其目的是为了满足不同用户和应用程序的需求。它可以帮助企业和组织构建完整的解决方案,提高效率和降低成本。
    
    为何需要构建服务生态系统:
    \begin{itemize}
        \item 一方面,面向服务的快速发展导致单个组织无法独立提供全套服务,提供的有限服务也无法被广泛运用;已存在的服务并不能很好的被发现和调用,也导致了大量冗余服务
        \item 另一方面,原先的服务系统是复杂、脆弱、特殊的,从上层业务看,无法灵活应对实际业务的变更;从底层实现看,也无法及时应对底层技术的更新、或者新增的功能
        \item 因此构建服务生态系统,运用面向服务的分析和设计原则,使得产生的服务具有良好的可发现性和可复用性,同时能灵活应对业务领域和技术领域的变更
    \end{itemize}
    
    \begin{spacing}{1.2}
        \vspace{-0.5em}
        \begin{longtable}{|m{7.5cm}|m{7.5cm}|}
            \hline
            \multicolumn{1}{|c|}{\textbf{垂直服务}} & \multicolumn{1}{c|}{\textbf{水平服务}} \\ \hline
            \vspace{-1.3em}
            \begin{itemize}[leftmargin=1.5em,itemsep=-3pt]
                \item 单独面向一个客户,提供系列功能的服务
                \item 从消费者的角度说,垂直服务可以被同时、独立地使用,分为纯IT服务和IT使能的服务
            \vspace{-1.5em}
            \end{itemize}                                           
                & 
            \vspace{-1.3em}
            \begin{itemize}[leftmargin=1.5em,itemsep=-3pt]
                \item 用于构建垂直服务、可重用的、跨行业的公共服务
                \item 分为公共业务服务和IT服务
            \vspace{-1.5em}
            \end{itemize}  
            \\ \hline
        \end{longtable}
        \vspace{-1em}
    \end{spacing}
    
    
    垂直服务和水平服务的联系与区别:
    \begin{itemize}
        \item 垂直服务和水平服务都是通过服务系统来实现业务服务
        \item 水平服务是功能相关的,简单且相对稳定,一般由IT专家开发
        \item 垂直服务是流程相关的,复杂且易变更,需要领域专家参与开发
        \item 垂直服务实际上是一系列水平服务的封装,将上下文无关的水平服务根据特定的业务流程进行编排,最后打包为一个解决特定问题的垂直服务
    \end{itemize}
    
    举例:
    学校的借书服务、打印成绩单服务是垂直服务,都需要进行身份认证,身份认证就是一个水平服务;借书服务和打印成绩单服务除了身份认证还包含其他的服务,就是将简单的水平服务构建成能实现业务目标或流程的垂直服务
    \end{solution}