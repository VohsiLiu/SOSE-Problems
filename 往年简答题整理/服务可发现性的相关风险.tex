\begin{longtable}{|m{3cm}<{\centering}|m{12cm}|}
    \hline
    可发现性在实施后的应用
    & 
    \vspace{-1.3em}
    \begin{itemize}[leftmargin=1.5em,itemsep=-3pt,topsep=-3pt]
        \item 在服务定义完毕后,再记录元数据,甚至由其他人员来加以记录,从而导致发现性和可解释性元数据的质量的损失
        \item 应当在设计阶段,早于服务最初发布时,就把那些元信息添加到文档中 
    \vspace{-1.5em}
    \end{itemize}  
    \\ \hline
    由不擅交流的人员来应用本原则
    & 
    \vspace{-1.3em}
    \begin{itemize}[leftmargin=1.5em,itemsep=-3pt,topsep=-3pt]
        \item 如果可发现性信息仅仅是由业务或者技术专家创建的,那么它很可能不足以应付其他的项目组成员的使用
    \vspace{-1.5em}
    \end{itemize}  
    \\ \hline
\end{longtable}
