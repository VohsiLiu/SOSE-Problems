\begin{problem}
    制造与服务的区别?
    \end{problem}
    
    \begin{solution}
    \begin{spacing}{1.25}
        \begin{longtable}{|W{c}{1.8cm}|m{6.4cm}|m{6.2cm}|}
            \hline
            \multicolumn{1}{|c|}{\textbf{}} & \multicolumn{1}{c|}{\textbf{制造}} & \multicolumn{1}{c|}{\textbf{服务}} \\ \hline
            物质性 & 产品是可以被实体触摸的有形物品 & 服务是无形的,不能被实体触摸 \\ \hline
            交付方式 & 产品通常以物理形式交付 & 服务通过可远程访问的接口交付 \\ \hline
            寿命 & 产品有一个固定的寿命,通常使用到磨损或过时 & 服务可以重复使用,并且随着时间的推移可能会被更新或修改 \\ \hline
            定制 & 产品通常是大量生产的,不能轻易定制 & 服务可以根据客户的需求进行定制 \\ \hline
            价值 & 产品通常基于其特征和质量进行估值 & 服务则基于其满足特定需求或解决特定问题的能力进行估值 \\ \hline
        \end{longtable}
        \vspace{-1em}
    \end{spacing}
    
    商品与服务的精确定义应根据其属性加以区分
    \begin{itemize}
        \item 商品是可以创造和转让的有形实物或产品;它随着时间的推移而存在,因此可以在以后创建和使用。
        \item 服务是无形的,易变质的。它是同时或几乎同时创建和使用的事件或过程。虽然消费者在实际服务产生后无法保留,但服务的努力可以保留。
    \end{itemize}
    \end{solution}