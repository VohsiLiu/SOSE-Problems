\begin{problem}
XML的结构是什么?与原先的文本/json/二进制相比有什么好处?
\end{problem}

\begin{solution}
XML是一种用于描述数据的标记语言,形成的是树状结构,一个XML文档有唯一的element作为根元素,该元素是所有其他元素的父元素,同时所有的元素均可拥有子元素。XML中的元素由一个开始标签、一个结束标签和其中包含的内容组成。开始标签和结束标签之间的内容可以包含其他元素、文本和属性。

相比于文本、JSON和二进制格式,XML有以下几个优点:
\begin{enumerate}[label=\arabic*.]
    \item 可读性:XML的标记结构和层次关系非常清晰,易于理解和阅读,尤其是对于人类来说。这使得XML非常适合用于传输和存储结构化数据。
    \item XML文档的内容和结构完全分离,基于这个特性可以实现服务的功能管理和流程管理彻底分离。
    \item 互操作性:纯文本文件可以方便地在不同的系统之间通信。
    \item 跨平台性:XML是一种独立于平台和语言的标记语言,可以在不同的操作系统、编程语言和应用程序之间进行交互和传输。
    \item 可扩展性:XML是一种可扩展的标记语言,可以根据需要自定义标记和元素,以适应各种数据结构的需求。
    \item 支持多语言:XML可以支持多种语言和字符集,包括Unicode和其他字符编码,方便多语言系统对数据进行处理。
\end{enumerate}

\end{solution}