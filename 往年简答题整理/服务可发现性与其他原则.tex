\begin{longtable}{|m{3cm}<{\centering}|m{12cm}|}
    \hline
    服务可发现性与标准化服务合约
    & 
    \vspace{-1.3em}
    \begin{itemize}[leftmargin=1.5em,itemsep=-3pt,topsep=-3pt]
        \item 使服务更加容易可发现和可解释会影响服务合约的内容
        \item 服务可发现性会直接地影响功能表达设计标准的确定 
    \vspace{-1.5em}
    \end{itemize}  
    \\ \hline
    服务可发现性与服务抽象
    & 
    \vspace{-1.3em}
    \begin{itemize}[leftmargin=1.5em,itemsep=-3pt,topsep=-3pt]
        \item 服务抽象的原则需要减少合约当中所发布的信息数量;服务可发现性则要求提供更多的信息;两者之间需要取得平衡
        \item 一旦实现了可发现性和抽象之间的适当的平衡,那么随后实现的服务的可发现性将基于那些已发布的(而不是被抽象的)元信息 
    \vspace{-1.5em}
    \end{itemize}  
    \\ \hline
    服务可发现性与服务可复用性
    & 
    \vspace{-1.3em}
    \begin{itemize}[leftmargin=1.5em,itemsep=-3pt,topsep=-3pt]
        \item 强调服务可发现性的主要目的是支持服务可复用性
        \item 当表述可复用功能时,应当应用可发现性相关的设计标准,以保证能通过实际的技术合约把服务的目的和能力尽可能清楚地表述出来 
    \vspace{-1.5em}
    \end{itemize}  
    \\ \hline
    服务可发现性与服务可组合性
    & 
    \vspace{-1.3em}
    \begin{itemize}[leftmargin=1.5em,itemsep=-3pt,topsep=-3pt]
        \item 潜在的组合成员应当容易定位和识别,以避免在无意间创建冗余的服务逻辑
        \item 当服务组合为了适应上层业务流程的变化或者为了增加整体的业务需求实现而发生演变时,需要查找从组合的原始版本创建以来,新加入的服务和功能 
    \vspace{-1.5em}
    \end{itemize}  
    \\ \hline
\end{longtable}
