\begin{longtable}{|m{3cm}<{\centering}|m{12cm}|}
    \hline
    服务可复用性与标准化服务合约
    & 
    \vspace{-1.3em}
    \begin{itemize}[leftmargin=1.5em,itemsep=-3pt,topsep=-3pt]
        \item 可复用的服务需要足够的灵活性来支持带有不同交互需求的消费者
        \item 导致降低合约验证约束(尤其是那些易变的)的设计标准
    \vspace{-1.5em}
    \end{itemize}  
    \\ \hline
    服务可复用性与服务抽象
    & 
    \vspace{-1.3em}
    \begin{itemize}[leftmargin=1.5em,itemsep=-3pt,topsep=-3pt]
        \item 合约的自描述性与简洁之间的平衡
        \item 元信息的抽象程度反映这一平衡 
    \vspace{-1.5em}
    \end{itemize}  
    \\ \hline
    服务可复用性与服务松散耦合
    & 
    \vspace{-1.3em}
    \begin{itemize}[leftmargin=1.5em,itemsep=-3pt,topsep=-3pt]
        \item 一个服务的依赖需求越小,复用它就越简单
        \item 当追求服务逻辑的可复用性时,总是有一种减少服务合约约束的趋势
    \vspace{-1.5em}
    \end{itemize}  
    \\ \hline
    服务可复用性与其他原则
    & 
    \vspace{-1.3em}
    \begin{itemize}[leftmargin=1.5em,itemsep=-3pt,topsep=-3pt]
        \item 服务自治:自治是对可复用服务潜在高性能和并行使用的保证
        \item 服务无状态:通过最小化状态管理责任,提高一个服务的可用性,从而提高有效扩展的能力
        \item 服务可发现性:可复用服务必需可发现、可解释
        \item 服务可组合性:可组合是复用的一种形式,可复用潜能越大,服务被反复组装的机会就越大 
    \vspace{-1.5em}
    \end{itemize}  
    \\ \hline
\end{longtable}