\begin{longtable}{|m{3cm}<{\centering}|m{12cm}|}
    \hline
    版本化(服务合约的演化)
    & 
    \vspace{-1.3em}
    \begin{itemize}[leftmargin=1.5em,itemsep=-3pt,topsep=-3pt]
        \item 服务实施后,就有可能建立起与服务消费者之间的依赖关系
        \item 底层逻辑越是可复用,那些需要消费它的程序的数量和消费频率就会越大
        \item 可扩展性可能引入“破坏”既定合约的重大变化,从而导致发布新的服务版本的要求 
    \vspace{-1.5em}
    \end{itemize}  
    \\ \hline
    技术依赖
    & 
    \vspace{-1.3em}
    \begin{itemize}[leftmargin=1.5em,itemsep=-3pt,topsep=-3pt]
        \item 不同的编程语言和开发平台来实现
        \begin{itemize}[leftmargin=1.5em,itemsep=-3pt,topsep=-3pt]
            \item 使用基于构件的系统并通过增强的RPC技术以支持面向服务
            \item Web Service平台以及它的非专用的通信框架
        \end{itemize} 
        \item 操作性系统层的技术性变化导致服务合约变化
    \vspace{-1.5em}
    \end{itemize}  
    \\ \hline
    开发工具缺陷
    & 
    \vspace{-1.3em}
    \begin{itemize}[leftmargin=1.5em,itemsep=-3pt,topsep=-3pt]
        \item 使用开发工具自动生成合约可能产生非标准化的服务合约
    \vspace{-1.5em}
    \end{itemize}  
    \\ \hline
\end{longtable}