\begin{longtable}{|m{3cm}<{\centering}|m{12cm}|}
    \hline
    错误地判断服务的范围
    & 
    \vspace{-1.3em}
    \begin{itemize}[leftmargin=1.5em,itemsep=-3pt,topsep=-3pt]
        \item 服务自治倾向于产生隔离的服务
        \item 如果在服务建模过程中,错误计算或者错误判断了服务的范围定义,在被部署到一个隔离的环境之后,将很难再被改变
        \item 服务能力也有类似的问题
    \vspace{-1.5em}
    \end{itemize}  
    \\ \hline
    包装服务和遗留逻辑封装
    & 
    \vspace{-1.3em}
    \begin{itemize}[leftmargin=1.5em,itemsep=-3pt,topsep=-3pt]
        \item 与封装相关联的服务自治风险包括:
        \begin{itemize}[leftmargin=1.5em,itemsep=-3pt,topsep=-3pt]
            \item 为了实现服务而使用的服务适配器不够灵活,并且无法被充分定制。这会威胁到标准化和可发现性这类设计特性
            \item 底层遗留环境是无法被定制的,因而会危害到其他面向服务原则的应用
        \end{itemize} 
        \item 当服务的实现需要封装遗留逻辑时,它所能达到的自治级别几乎总是会明显降低
    \vspace{-1.5em}
    \end{itemize}  
    \\ \hline
    对服务需求的过高估计
    & 
    \vspace{-1.3em}
    \begin{itemize}[leftmargin=1.5em,itemsep=-3pt,topsep=-3pt]
        \item 服务自治要求较高的资源,长期高估服务的使用需求,可能会在一定程度上破坏面向服务计算的减低IT负载的目标
        \item 因此,即使资源非常充裕,对需要达到较高自治级别的服务也需逐个进行代价评估
    \vspace{-1.5em}
    \end{itemize}  
    \\ \hline
\end{longtable}