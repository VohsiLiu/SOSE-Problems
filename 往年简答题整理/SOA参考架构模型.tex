\subsubsection*{SOA参考架构}
\begin{figure}[H]
    \vspace{-0.5em}
	\centering
	\includegraphics[width=0.7\textwidth]{SOA参考架构.pdf}
    \vspace{-1em}
\end{figure}

水平层:对功能性需求加以满足,五个水平层分为服务提供者和服务消费者两组:
\begin{itemize}
    \item 服务提供者(后台)
\end{itemize}
\begin{spacing}{1.2}
    \input{SOA水平层服务提供者.tex}
    \vspace{-2.5em}
\end{spacing}

\begin{itemize}
    \item 服务消费者(前台)
\end{itemize}

\begin{spacing}{1.15}
    \input{SOA水平层服务消费者.tex}
    \vspace{-1em}
\end{spacing}

垂直层:对当前系统进行支撑以及实现服务质量、非功能性需求
\begin{spacing}{1.15}
    \input{SOA垂直层.tex}
    \vspace{-2em}
\end{spacing}

\begin{itemize}
    \item SOA-RA 展示了如何将 SOA 解决方案构建为一组逻辑层的抽象。
    \item SOA-RA 是一种松耦合的架构,因为每个层不严格隐藏在上面的层之中。
    \item SOA-RA 是一个企业级架构模板,通过定义参考架构指导在企业级别上创建 SOA 解决方案。
    应用 SOA-RA 模型来定义 SOA 导向的系统架构的一种实践称为服务导向建模和架构(SOMA)。
    \item 在SOA-RA层中配置组件定义了三个步骤:
    \vspace{-0.8em}
    \begin{multicols}{3}
    \begin{itemize}
        \item 服务识别步骤
        \item 服务规范步骤
        \item 服务实现步骤
    \end{itemize}
    \end{multicols}
    \vspace{-1em}
\end{itemize}