\begin{longtable}{|m{3cm}<{\centering}|m{12cm}|}
    \hline
    服务自治与标准化服务合约
    & 
    \vspace{-1.3em}
    \begin{itemize}[leftmargin=1.5em,itemsep=-3pt,topsep=-3pt]
        \item 服务合约自治直接与服务合约紧密相连
        \item 规范化的考虑会影响到合约如何形成,以及如何与其他服务协调
        \item 在服务合约上有越大的控制权,服务合约能被更好地定制和标准化,越能够确保底层实现可以在遵循既定自治级别的前提下,被独立设计
    \vspace{-1.5em}
    \end{itemize}  
    \\ \hline
    服务自治与服务松散耦合
    & 
    \vspace{-1.3em}
    \begin{itemize}[leftmargin=1.5em,itemsep=-3pt,topsep=-3pt]
        \item 由于同样期望将服务之间的依赖最小化,服务自治在很大程度上支持服务松散耦合原则:积极耦合会直接导致设计时自治的增加;设计时自治的增加,又能更好地增强和优化服务的实现,从而支持运行时的自治
    \vspace{-1.5em}
    \end{itemize}  
    \\ \hline
    服务自治与服务抽象
    & 
    \vspace{-1.3em}
    \begin{itemize}[leftmargin=1.5em,itemsep=-3pt,topsep=-3pt]
        \item 将一个服务的自治级别作为整个服务合约的一部分来发布
        \item 服务自治的信息是服务质量信息抽象的一个例子
    \vspace{-1.5em}
    \end{itemize}  
    \\ \hline
    服务自治与服务可复用性
    & 
    \vspace{-1.3em}
    \begin{itemize}[leftmargin=1.5em,itemsep=-3pt,topsep=-3pt]
        \item 自治的增加提高了一个服务的复用潜力
        \item 通过增强服务的可靠性和提高服务行为的可预测性,其逻辑可以更加容易地适应多个服务消费者的需求
        \item 更好地支持服务运行环境的演化,从而应对复用所带来的并发要求 
    \vspace{-1.5em}
    \end{itemize}  
    \\ \hline
    服务自治与服务无状态性
    & 
    \vspace{-1.3em}
    \begin{itemize}[leftmargin=1.5em,itemsep=-3pt,topsep=-3pt]
        \item 实现高级别的服务自治可以直接支持服务无状态性程度的增加
    \vspace{-1.5em}
    \end{itemize}  
    \\ \hline
    服务自治与服务可组合性
    & 
    \vspace{-1.3em}
    \begin{itemize}[leftmargin=1.5em,itemsep=-3pt,topsep=-3pt]
        \item 服务组合的整体自治性取决于它的所有组成成员自身的自治性
        \item 服务有越好的可靠性和可预侧性就越能组成更高效的大型服务组合 
    \vspace{-1.5em}
    \end{itemize}  
    \\ \hline
\end{longtable}