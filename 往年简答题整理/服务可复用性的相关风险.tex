\begin{longtable}{|m{3cm}<{\centering}|m{12cm}|}
    \hline
    文化上的考虑
    & 
    \vspace{-1.3em}
    \begin{itemize}[leftmargin=1.5em,itemsep=-3pt,topsep=-3pt]
        \item 程序员和架构师不愿意去使用
    \vspace{-1.5em}
    \end{itemize}  
    \\ \hline
    治理上的考虑
    & 
    \vspace{-1.3em}
    \begin{itemize}[leftmargin=1.5em,itemsep=-3pt,topsep=-3pt]
        \item 面向服务将相互无关的逻辑单元抽象为服务,与业务流程、应用程序或用户基础都没有任何直接联系
    \vspace{-1.5em}
    \end{itemize}  
    \\ \hline
    可靠性上的考虑
    & 
    \vspace{-1.3em}
    \begin{itemize}[leftmargin=1.5em,itemsep=-3pt,topsep=-3pt]
        \item 可复用服务的单点失效会导致多个业务流程的失败
        \item 通过对关键服务的多重复用来解决
    \vspace{-1.5em}
    \end{itemize}  
    \\ \hline
    安全上的考虑
    & 
    \vspace{-1.3em}
    \begin{itemize}[leftmargin=1.5em,itemsep=-3pt,topsep=-3pt]
        \item 在不同应用场景中的安全性要求不同
        \item 安全级别可能和信息交换的方式直接相关,甚至可能和服务合约所暴露的功能类型相关
    \vspace{-1.5em}
    \end{itemize}  
    \\ \hline
    商业设计需求上的考虑
    & 
    \vspace{-1.3em}
    \begin{itemize}[leftmargin=1.5em,itemsep=-3pt,topsep=-3pt]
        \item 领域专家在进行服务分析和建模阶段中引入的风险和问题
    \vspace{-1.5em}
    \end{itemize}  
    \\ \hline
    敏捷交付上的考虑
    & 
    \vspace{-1.3em}
    \begin{itemize}[leftmargin=1.5em,itemsep=-3pt,topsep=-3pt]
        \item 在需要以敏捷开发方法来解决短期和战术上的业务目标时,提倡服务的可复用性是非常困难的
    \vspace{-1.5em}
    \end{itemize}  
    \\ \hline
\end{longtable}