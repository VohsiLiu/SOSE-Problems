\begin{longtable}{|m{3cm}<{\centering}|m{12cm}|}
    \hline
    服务松散耦合与标准化服务合约
    & 
    \vspace{-1.3em}
    \begin{itemize}[leftmargin=1.5em,itemsep=-3pt,topsep=-3pt]
        \item 松散耦合鼓励调节技术合约内容的数量和复杂度,从而最小化消费者依赖需求、最大化服务所有者的自由度,在不影响现有消费者的情况下随着时间演化和改变服务
    \vspace{-1.5em}
    \end{itemize}  
    \\ \hline
    服务松散耦合与服务抽象
    & 
    \vspace{-1.3em}
    \begin{itemize}[leftmargin=1.5em,itemsep=-3pt,topsep=-3pt]
        \item 创建更低耦合的消费者关系,明确地要求应用良好定义的功能和技术抽象级别
    \vspace{-1.5em}
    \end{itemize}  
    \\ \hline
    服务松散耦合与服务可复用性
    & 
    \vspace{-1.3em}
    \begin{itemize}[leftmargin=1.5em,itemsep=-3pt,topsep=-3pt]
        \item 减少依赖关系可以使服务更容易被组合、演化甚至扩充以支持不断变化的业务需求和方向
    \vspace{-1.5em}
    \end{itemize}  
    \\ \hline
    服务松散耦合与服务自治
    & 
    \vspace{-1.3em}
    \begin{itemize}[leftmargin=1.5em,itemsep=-3pt,topsep=-3pt]
        \item 减少消极耦合类型的程度,会为运行时和设计时的更高自治级别提供支持
        \item 服务消费者具有越多的跨服务依赖,它所具有的自主权就越少(服务消费者可能同时担任复合服务中的服务协调者)
    \vspace{-1.5em}
    \end{itemize}  
    \\ \hline
    服务松散耦合与服务可发现性
    & 
    \vspace{-1.3em}
    \begin{itemize}[leftmargin=1.5em,itemsep=-3pt,topsep=-3pt]
        \item 服务松散耦合有助于元数据的调节
    \vspace{-1.5em}
    \end{itemize}  
    \\ \hline
    服务松散耦合与服务可组合性
    & 
    \vspace{-1.3em}
    \begin{itemize}[leftmargin=1.5em,itemsep=-3pt,topsep=-3pt]
        \item 在服务组合中,避免消极形式的耦合
        \begin{itemize}[leftmargin=1.5em,itemsep=-3pt,topsep=-3pt]
            \item “合约-逻辑”耦合:如果服务合约是自动生成的,就很有可能在被其他服务使用时不符合标准。因此需要在它和其他组成成员之间进行转换
            \item “合约-技术”耦合:如果同一个组合中的不同部分同时使用开放与专用服务技术,就会需要在本地实现技术转化层
            \item “合约-实现”耦合:当一个服务合约与底层实现特性之间产生耦合时,就会最终把这些性质强加到作为一个整体的组合之上
        \end{itemize} 
    \vspace{-1.2em}
    \end{itemize}  
    \\ \hline
\end{longtable}