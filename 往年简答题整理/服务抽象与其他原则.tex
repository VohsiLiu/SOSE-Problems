\begin{longtable}{|m{3cm}<{\centering}|m{12cm}|}
    \hline
    服务抽象与标准化服务合约
    & 
    \vspace{-1.3em}
    \begin{itemize}[leftmargin=1.5em,itemsep=-3pt,topsep=-3pt]
        \item 服务抽象出来并对外界可用的信息就是服务合约,服务抽象原则的应用影响到服务合约
        \item 服务合约的设计标准也会影响到功能、技术和逻辑抽象的等级 
    \vspace{-1.5em}
    \end{itemize}  
    \\ \hline
    服务抽象与服务松散耦合
    & 
    \vspace{-1.3em}
    \begin{itemize}[leftmargin=1.5em,itemsep=-3pt,topsep=-3pt]
        \item 抽象的程度对可能耦合的程度有直接的关系
        \item 少量的高度详细的技术接口约束会导致比大量含糊或开放的数据约束更多的紧密耦合需求
        \begin{itemize}[leftmargin=1.5em,itemsep=-3pt,topsep=-3pt]
            \item 耦合的程度一般由被抽象的信息数量和信息本身的属性的组合来决定
            \item 最终由服务合约的粒度加以体现 
        \end{itemize}
    \vspace{-1.2em}
    \end{itemize}  
    \\ \hline
    服务抽象与其他原则
    & 
    \vspace{-1.3em}
    \begin{itemize}[leftmargin=1.5em,itemsep=-3pt,topsep=-3pt]
        \item 其他的服务设计原则,如服务可复用性、服务可组合性和服务可发现性等原则都鼓励创建更多的、关于服务的元信息
        \item 而服务的抽象原则要求在发布这些元信息前评估其必要程度
    \vspace{-1.5em}
    \end{itemize}  
    \\ \hline
\end{longtable}