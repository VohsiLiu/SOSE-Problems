\begin{problem}
以电信企业为应用背景,举例描述服务分析和服务设计的过程。并结合面向服务的设计原则(标准化服务合约、服务松散耦合、服务抽象、服务可复用性、服务自治、服务无状态性、服务可发现性、服务可组合性),讨论“schema集中化”“合约集中化”“逻辑集中化”在设计过程中的应用。
\end{problem}

\begin{solution}
\textbf{服务分析的过程} \par
面向服务的分析的目标是讨论需要构建哪些服务,每个服务需要封装哪些逻辑
\begin{enumerate}[label=\arabic*.]
    \item 定义流程自动化需求,形成文档化的需求描述,这是服务候选建模的依据
    \item 识别现有的自动化系统,分析企业正在使用的系统具有的功能
    \item 对服务候选建模,识别服务操作候选,并将其分组到服务候选中
\end{enumerate}

在面向服务分析流程中,需要考虑服务可复用性、服务自治和服务可发现性
\begin{itemize}
    \item 可复用性:在服务建模中,需要:精化已有的服务能力候选,使其更加一般化和可复用;定义额外的服务能力候选,这些能力是在构成服务建模过程的基础的业务流程自动化所需之外的
    \item 自治:对已有自动化系统收集得到的信息,会影响服务系统所能达到的自治级别;比如根据信息決定保留遗留系统,那么达到共享自治,独立开发的可能达到逻辑自治或完全自治
    \item 可发现性:从服务生命周期开始,尤其是在产生服务操作候选时,需要以统一的方式,记录所有元数据;在服务建模过程中,业务和技术专家需要一起合作,建立服务候选
\end{itemize}

\textbf{服务设计的过程} \par
服务设计过程,是从服务候选(逻辑)派生出具体的服务设计 (物理),然后装配到实现业务流程的抽象组合中。
\begin{enumerate}[label=\arabic*.]
    \item 组合SOA:选择编排服务层、业务服务层、应用服务层中的哪些进行实现,定义核心的SOA标准,选择SOA扩展(WS-*协议)
    \item 根据业务层级,分别设计以实体为核心的业务服务,应用服务,以任务为核心的业务服务
    \item 设计面向服务业务过程,组合服务构建出业务流程
\end{enumerate}

\textbf{“Schema集中化”} \par
传统的做法是在订购服务、退订服务中使用不同的套餐数据结构,而按照标准化服务合约,所有使用的数据结构都应该被单独定义、管理,与具体的操作流程无关。采用Schema集中化的设计模式,将电信企业划分为多个分离的领域(部门),每个领域都可以被独立地进行标准化和治理,每个领域定义和管理自己的Schema,作为整个服务系统的基本数据结构;在不同的服务中,使用这些Schema,避免了频繁且不必要的数据转换;在必要的情况下,可以利用这些Schema定义新的数据结构。

例如,在电信企业中,可以定义一个名为“用户信息”的Schema, 包括用户ID、姓名、电话号码、地址等信息。然后,各个服务都可以使用这个Schema来定义输入和输出参数,确保它们的数据格式是一致的,这样可以避免数据格式不统一导致的问题。

\textbf{“合约集中化”} \par
为了保证服务松散耦合,避免消极耦合,采用合约集中化,将对服务的访问严格控制在合约内:
\begin{itemize}
    \item 所有的合约应该被集中管理,拥有一致的设计原则和设计目标
    \item 在服务生态系统中,任何情况都不可以绕开合约去访问具体内容
\end{itemize}

合约集中化要求,设计者建立消费者程序时,这些消费者程序仅通过已发布的合约访问一个服务。

为了实现合约集中化,要根据服务抽象、服务可发现性原则设计服务暴露的信息
\begin{itemize}
    \item 服务抽象:技术信息、功能、程序逻辑、服务质量抽象
    \item 服务抽象出来并对外界可用的信息就是服务合约,服务合约的设计标准会影响到又会影响到其他因素
\end{itemize}

例如,在电信企业中,可以定义一个名为“账单查询”的服务合约,其中包括输入参数、输出参数、调用方式和错误处理等信息。各个服务都可以使用这个服务合约来定义自己的接口,确保接口的一致性和兼容性,这样可以避免因为接口不兼容导致的问题。

\textbf{“逻辑集中化”} \par
为了实现服务可复用性,让消费者程序只调用指定的服务,要建立服务库存,在规范的服务库存中,每个服务代表来一个独特的功能域,这就要求服务边界之问没有重叠。

也就是说,当需要某个服务时,首先应该查询服务库存中是否存在已有服务,若有则直接调用;否则应按照面向服务设计原则进行开发并补全到当前服务库存中。

逻辑集中化要求设计者在建立消费者程序并需要特定类型的信息处理时,这些消费者程序只调用指定的服务。

为了实现逻辑集中化,应按照标准化服务合约原则将所有的服务都按照统一的标准进行描述,最大程度的减少调用者无法发现现有服务的可能,进而提高服务的可发现性,从而实现可复用性。服务可发现性是实现服务可复用的前提,服务自治是可复用服务潜在高性能和并行使用的保证;无状态性能提高服务的可用性。

例如,在电信企业中,可以将“账单生成”和“账单发送”两个功能集中在一个名为“账单管理”的服务中实现,然后其他服务可以调用这个服务来生成和发送账单,而不需要每个服务都实现这些功能。这样可以减少代码冗余和维护成本,提高服务的可复用性和可维护性。


为了实现逻辑集中化和合约集中化,Web服务的WSDL、XML schema 和WS-Policy定义必须正确地表达访问一个正式逻辑体(按逻辑集中化)的一个正式访问点(根据合约集中化)。也就是服务消费者调用同一个功能的同一个服务,通过同一个接口来访问后台的实现。
\end{solution}