\begin{longtable}{|m{3cm}<{\centering}|m{12cm}|}
    \hline
    多消费者耦合的需求
    & 
    \vspace{-1.3em}
    \begin{itemize}[leftmargin=1.5em,itemsep=-3pt,topsep=-3pt]
        \item 不同消费者可能需要不同的技术接口细节,所需的抽象程度也不尽相同
        \item 使用合约反规范化,提供不同级别的抽象粒度
    \vspace{-1.5em}
    \end{itemize}  
    \\ \hline
    人为误判
    & 
    \vspace{-1.3em}
    \begin{itemize}[leftmargin=1.5em,itemsep=-3pt,topsep=-3pt]
        \item 过于抽象的服务合约导致曲解或不能充分理解一个服务。从而丧失潜在的复用机会
        \item 过于具体的服务合约导致对服务的行为作出与服务实现相关的假设,从而导致实现耦合
    \vspace{-1.5em}
    \end{itemize}  
    \\ \hline
    安全和隐私的考虑
    & 
    \vspace{-1.3em}
    \begin{itemize}[leftmargin=1.5em,itemsep=-3pt,topsep=-3pt]
        \item 服务合约可能暴露私有或者敏感信息
        \begin{itemize}[leftmargin=1.5em,itemsep=-3pt,topsep=-3pt]
            \item 不同的使用环境(组织内部 vs. 组织外部)
            \item 并发和冗余的服务合约内容以解决这一问题
        \end{itemize}
    \vspace{-1.2em}
    \end{itemize}  
    \\ \hline
\end{longtable}