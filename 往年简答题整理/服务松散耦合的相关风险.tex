\begin{longtable}{|m{3cm}<{\centering}|m{12cm}|}
    \hline
    “逻辑-合约”耦合的限制
    & 
    \vspace{-1.3em}
    \begin{itemize}[leftmargin=1.5em,itemsep=-3pt,topsep=-3pt]
        \item 同一底层逻辑对应两个或者多个合约,从而建立多个入口,每一入口向不同类型的消费者暴露不同的服务能力
    \vspace{-1.5em}
    \end{itemize}  
    \\ \hline
    Schema耦合太“松散”
    & 
    \vspace{-1.3em}
    \begin{itemize}[leftmargin=1.5em,itemsep=-3pt,topsep=-3pt]
        \item 为了强调服务的兼容性演化能力,通过过分简化服务合约,追求减少消费者依赖,仅确定了一些非常通用的数据类型(弱类型)
        \item 验证并处理弱类型,增加服务所需的性能要求
        \item 服务合约发布的信息越少,消费者程序就需要知道越多关于服务实现逻辑的信息,从而产生消极耦合  
    \vspace{-1.5em}
    \end{itemize}  
    \\ \hline
\end{longtable}