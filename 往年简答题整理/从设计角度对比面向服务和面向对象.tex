\vspace{-0.5em}
\begin{spacing}{1.2}
    \centering
    \begin{longtable}{|m{1.2cm}<{\centering}|m{6.7cm}|m{6.7cm}|}
        \hline
\textbf{特点} & \multicolumn{1}{c|}{\textbf{面向对象的计算}} & \multicolumn{1}{c|}{\textbf{面向服务计算}} \\ \hline
耦合          & 提倡重用和松耦合,但是预先定义的类依赖导致更多的对象紧密绑定        & 服务的松耦合由功能和服务合约给定                     \\ \hline
粒度          & 为支持不同规模的任务,支持细粒度接口(API)               & 鼓励粗粒度的接口(服务描述),通讯消息中包含尽可能多的任务相关信息    \\ \hline
作用域         & 对象作用域更小,更有针对性(往往基于一个软件系统)             & 服务作用域显著不同(往往基于一个服务生态系统)              \\ \hline
前瞻性         & 鼓励处理逻辑与数据的绑定从而产生对象                    & 鼓励创建活动无关的、由消息驱动的服务                  \\ \hline
状态性         & 数据和逻辑的绑定,导致带状态的对象                     & 服务尽可能保持无状态性                          \\ \hline
组合          & 在支持对象组合的同时也支持对象的继承,从而导致紧耦合            & 支持松散耦合服务的组合                          \\ \hline
    \end{longtable}
\end{spacing}
\vspace{-0.5em}